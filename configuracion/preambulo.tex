\usepackage[spanish, es-tabla]{babel}

% Permite la codificación UTF-8 para caracteres
\usepackage[utf8]{inputenc}

% Permite insertar figuras usando \placein{figure} en este punto
\usepackage[section]{placeins}

% Configura biblatex para el estilo APA con ordenación por autor, citas nombre-año, backend biber
\usepackage[style=apa,sortcites=true,sorting=nyt,backend=biber]{biblatex}
\bibliography{contenido/4.Referencias/bibliografia.bib}

% Carga el paquete csquotes para un manejo inteligente de comillas
\usepackage{csquotes}

% Carga varios paquetes para tablas, texto verbatim, gráficos y URLs
\usepackage{array,fancyvrb,graphicx,verbatim,xurl}

% PARA GENERAR TEXTO RANDOM
\usepackage{lipsum}  

% (No cambiar)
\title{\titulo}
\shorttitle{\titulocorto}
\author{\autor}
\affiliation{\afiliacion}
